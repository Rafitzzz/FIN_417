We first start by exploring some basic notions of risk, and mainly define all of the financial jargon that is used throughout the course.

\subsection{Financial risk and regulation}
There are a lot of definitions of risk, but as for many things in life, no sentence can capture the full essence of a concept (especially in finance since all of the theory is constructed by phony, try-harding beaurocrats i.e. beaurocs).
As such, we will just list some of the most common definitions of risk:

\vspace{0.2cm}

\begin{itemize}
    \item The Concise Oxford English Dictionary: `hazard, a chance of bad consequences, loss or exposure to mischance'.
    \item The course textbook (M.F.E. McNeil, R. Frey, P. Embrechts, Quantitative Risk Management: Concepts, Techniques and Tools, Princeton University Press, 2015): `any event or action that may adversely affect an organization's ability to achieve 
    its objectives and execute its strategies'.
    \item ISO 31073: ` efect of uncertainty on objectives; can be positive, negative or both, can result in opportunities and threats'. This same guide also defines: 
    \begin{itemize}
        \item \textbf{Uncertainty}: state, even partial, of deficiency of information related to understanding or knowledge (`randomness').
        \item \textbf{Objective}: result to be achieved.
        \item \textbf{Risk management}: coordinated activities to direct and control an organization with regard to risk.
    \end{itemize}
\end{itemize}

\vspace{0.2cm}

As you can see, pure jargon and buzzwords are used to define risk, which is not very helpful. In this course we define risk as the following notion.

\begin{definition}
    Risk is the chance of financial loss due to uncertainty or `randomness' (but what is randomness anyway? don't even ask cuz' I get erected).
\end{definition}

So, given this defintion, we may model and measure risk using probabilistic (i.g. random variables, distributions) and statistical (i.g. data) tools, which is what we will do in this course. Mainly, we may thank daddy Kolmogorov for developing
a reliable mathematical framework (using probabilistic axioms) to think about risk in a rigorous manner. Since this course is heavily focused on finance, we will mainly be concerned with financial risk, which, yoy guessed it, has a shit ton of
variable definitions (each focusing on diferent aspects of finance).

\begin{remark}
    Note that financial risk is generated by a future change in value of an asset.
\end{remark}

\vspace{0.2cm}

Here are some of the main types of financial risk:

\begin{itemize}
    \item \textbf{Market risk}: change in value due to changes in underlying components (such as
    stock, bond, or commodity prices).
    \item \textbf{Credit risk}: possibility of not obtaining future payments due to fuck ups of the counterparty.
    \item \textbf{Operational risk}: risk of loss due to inadequate or failed internal process, people (beaurocs in action!) and systems  or from external events (fraud, earthquakes).
    \item \textbf{Liquidity risk}: not being able to buy or sell something quickly enoguh to minimize loss.
    \item \textbf{Model risk}: using an inadequate model for measuring risk.
\end{itemize}

\vspace{0.2cm}

Note that financial firms are not passive (i.e. defensive) towards risk, they confront it in order to gain return (risk tha biscuit!). Thus, Banks and insurers are like risk function engineers. They take raw random variables (risky outcomes), apply transformations $f(X)$,
 and then allocate these transformed versions to those most willing to hold them. That process is what keeps financial markets running smoothly.

 \begin{definition}
    A \textit{financial derivative} is a contract whose payoff is a function $f$ of some underlying random variable $U$ (e.g., stock price at time $T$). The whole industry is about analyzing, pricing, and hedging such contracts under uncertainty.
 \end{definition}